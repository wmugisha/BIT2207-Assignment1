\documentclass[options]{article}
\begin{document}
{\textbf{How to become a web designer.}}

\subsection{\textbf{AUTHOR DETAILS}}
 MUGISHA WILLIAM, 215012018, 15/U/7917/EVE

\section{\textbf{Introduction}}
Becoming a Web Designer is a cumulative process that builds up your skills day after day and year after year. The field of web design can be fun and rewarding (mentally, spiritually and financially) and this report gives a general outline of how to become a Web Designer
\section{\textbf{Know what Web programming entails. }}
Web applications are software components designed to work on top of the internet architecture. This means that the applications are accessed through a web browser software such as Firefox or Internet Explorer. Being built on top of the Internet architecture does not necessarily require an active connection to the internet. It means that Web applications are built on top of standard web technologies such as: HTTP, FTP, POP3 and much more.
\section{\textbf{Browse many diverse websites to learn about how they usually look. }}
(Right click, then click View Source or press F12.) Look for diversity in the type/content of the website, not the quantity of websites visited.
\section{\textbf{Learn at least one brainstorming technique/method and a software that is used to implement that method}}. For example: brainstorming diagrams and MS Visio.
\section{\textbf{Get familiar with website structuring. }}
This is creating conceptual web diagrams, site-maps, and navigation structures.  
\section{\textbf{Take a crash course on graphics design. }}
Try to learn at least one graphics editing/manipulation software package (optional, but strongly recommended)
\section{\textbf{Learn the basics of the internet infrastructure.}} 
This includes getting a basic idea about:
    Base Web services protocols (HTTP, FTP, SMTP, and POP3 or IMAP4)
    Web server software (preferably, one for the platform you will be working on mostly)
    Web browsing software. 
    Email server and client software.

\section{\textbf{Learn the HTML and CSS languages. }}
You might also want to get the "What You See Is What You Get (WYSIWYG)" software package for editing HTML


\section{\textbf{Learn XML and XML related technologies.}}
Create simple static websites until you are familiar with and comfortable around HTML.

\section{\textbf{Learn a client-side scripting language. }}
Most users learn JavaScript. Some learn VBScript, but this isn't compatible with most browsers.
\section{\textbf{Familiarize yourself with the client-side scripting language you learned. }}
Try to reach your potential using only that language. Only go to the next step after you've at least become familiar with your client-side scripting language.
11. Learn at least one server-side programming language. 
If you choose to restrict yourself to one server software, learn one of the programming languages supported by that software. If not, learn at least one programming language on each server software.
12. Create a pilot project for yourself after you finish learning the server-side programming language. You can obtain your own website and start experimenting online within your own page.
	\section{\textbf{Conclusion}}
Doing the above does not promise to give a magically easy way to becoming a Web programmer, and the ordering of the steps is not sacred, but you'll get a general outline of how to become a programmer in one of the modern programming fields
\section{\textbf{Recommendation}}
it’s no secret that earning requires repetition, it would therefore be wise to take on challenges that will help horn your web programming skills.

\section{\textbf{Reference}}
How to become a programmer? (2017). Retrieved from http://www.wikihow.com/Become-a-Programmer





\end{document}